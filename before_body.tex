\begin{romanpages}      % roman-numbered pages 

\TitlePage 

\doublespacing
\setlength{\parskip}{0pt plus 0pt minus 0pt}

\begin{abstract} 
\noindent
The management of natural resources is fraught with difficulties stemming from uncertainty and conflicting objectives. Fisheries for Pacific salmon \textit{Oncorhynchus} spp. in western Alaska are no exception, but rather provide a fantastic example. This area of the world is tremendously remote, making resource monitoring and decision-making challenging. It has widely been proposed that quantitative tools can be useful in aiding decision-making by producing predictions of uncertain states of nature, system responses to harvest, and quantifying likely outcomes of candidate management actions. In this dissertation, several such quantitative tools seeking to serve these purposes are developed and evaluated. Throughout, the Kuskokwim River subsistence salmon fishery (one of the largest in the world) is used as a case study to illustrate the development, evaluation, and application of these tools.

Before investigating three primary research topics in Chapters \ref{ch2}, \ref{ch3}, and \ref{ch4}, an overview of the difficulties faced by practitioners of salmon management is provided in Chapter \ref{ch1}. The management of salmon resources is presented as a three-tiered decision-making hierarchy made up of (1) fundamental objectives stemming from societal values, (2) inter-annual management strategies which define how the objectives are to be attained in the long-term, and (3) finer-scale (\textit{e}.\textit{g}., in-season) tactics used to implement the strategy. The important considerations and key uncertainties at play at each level are discussed, as is the role of quantitative tools in decision-making.

Chapter \ref{ch2} focuses on salmon migration timing, the problems that its inter-annual variability inserts for salmon managers and assessments of run abundance, and the development and evaluation of a forecasting tool that attempts to predict the timing of the run before any fish arrive. Run timing variability is a well-known and pervasive problem: in-season abundance index data are often consistent with many different run scenarios ranging from small and early to large and late, making decisions about the magnitude of harvestable surplus difficult. Run timing has been widely shown to covary with environmental variables linked to temperature such that early run years tend to coincide with warm years, suggesting that useful predictive relationships may exist. A statistically rigorous approach to forecasting the date of 50\% run completion for Kuskokwim River Chinook salmon \textit{O. tshawytscha} was developed based on relationships with temperature-related variables. An objective and intuitive temporal variable selection approach (the sliding climate window algorithm) was employed to determine the time periods for each variable most likely to produce accurate forecasts and the analysis relied heavily on multi-model inference in the face of a high degree of model uncertainty. The rather complex forecasting framework was found to perform no better on average at forecasting the median run date than the most naïve model that used solely the historical average as the forecast. Although the environmental variable forecast did not have value in terms of improving the accuracy of interpretations of an in-season run abundance index, its use did reduce the statistical uncertainty in this index, particularly early in the season. The analysis in this chapter should serve as a useful example for researchers faced with high-dimensional spatio-temporal variable selection problems, particularly with respect to the formal evaluation of the performance of alternative forecasts using retrospective cross-validation techniques.

Chapter \ref{ch3} takes a broader view of the in-season management problem by simulation-testing a set of four harvest control strategies in a framework known as stochastic management strategy evaluation. A detailed mathematical caricature of the Kuskokwim River salmon system was constructed based on empirical data to serve as the operating model with which to test different decision rules for determining how many days the fishery should be open each week. Strategy performance was assessed relative to four pre-defined management objectives dealing with both conservation and exploitation such that key trade-offs could be identified. A key research question was regarding strategy complexity and performance: whether more involved and data-intensive feedback strategies should be favored over simpler fixed-schedule strategies. The primary finding revealed by this work was that several strategies ranging from simple to complex can perform essentially equally well at attaining the objectives of salmon management in the Kuskokwim River as defined by the utility functions used. This finding suggests that in-season management may be made more difficult than needed, and that perhaps more focus should be placed on the transparency and defensibility of strategies.

Chapter \ref{ch4} takes an even broader view still, and addresses the topic of assessment of salmon fisheries that harvest fish from multiple substocks as a mixed-stock. Different substocks within a larger salmon-producing drainage vary in their size and resilience to harvest pressure (\textit{i}.\textit{e}., intrinsic productivity), which suggests a trade-off between maximizing harvest and preserving substock biodiversity exists. To incorporate this trade-off into harvest policies, an assessment of the heterogeneity in substock size and productivity must be completed. Salmon populations are often assessed as single stock units using simple regression approaches, however, this chapter makes a compelling case for integrating data from multiple substocks into a single assessment model that represents population dynamics and observation processes at the substock-level using a state-space analytical framework. A range of assessment methods were constructed that varied in their assumptions about process variability and were applied to empirical data from Kuskokwim River Chinook salmon substocks and evaluated in a simulation-estimation context. All state-space models assessed performed substantially better than regression-based approaches at returning accurate and precise estimates of biological reference points at the substock- and mixed-stock level, and by capturing the sources of variability allowed for more rich ecological interpretations which may be useful in future policy analyses. 

The dissertation concludes with Chapter \ref{ch5} which presents further reflection on the utility, performance, and generality of the tools developed in these studies.

\end{abstract}

\begin{acknowledgments}
\noindent
I am grateful to many people for helping to make my graduate career successful and enjoyable. First, I would like to thank my major professor, Dr. Matt Catalano, for his superb mentorship, providing and allowing me to pursue many challenging opportunities, and granting me freedom and flexibility in how I completed my work. I have had many other mentors as well: Dr. Lew Coggins, Dr. Mike Jones, Dr. Brendan Connors, Steve Fleischman, and Dr. Dan Gwinn have all been instrumental in shaping me as a quantitative scientist and I am immensely fortunate to consider them as such. In completing my academic and applied work, I have had the opportunity to work with many dedicated professionals: Dr. Troy Farmer, Nick Smith, Zach Liller, Janessa Esquible, Gary Decossas, Dr. Bill Bechtol, Aaron Moses, and Ken Stahlnecker; all of whom have made the experience enjoyable. I have received top-notch instruction and advice on my work while at Auburn; specifically, I would like to thank: Drs. Ash Abebe and Stephen Dobson (for their insights on Chapter \ref{ch2}), Dr. Conor McGowan (for his course on structured decision making and useful modeling techniques in that context), Dr. Todd Steury (for serving as an external university reader and for his excellent course introducing a wide range of statistical methods in ecology, which provided me with a solid foundation in linear modeling early in my graduate career) and Dr. Laurie Stevison (for her gentle but thorough introduction to Bash Shell programming, which allowed me to conduct the Chapter \ref{ch4} analysis on the Alabama High-Performance Computing [HPC] cluster). My work would not be possible without the dedication of the developers of software I used; I extend many thanks to the R Core Development Team for building a comprehensive and intuitive statistical environment, Dr. Martyn Plummer for developing JAGS, and the team at RStudio for their contributions in making working in R pleasurable in all aspects of quantitative research from the early exploratory stages to the publication of final products (\textit{e}.\textit{g}., this dissertation was written using RStudio's \texttt{\{bookdown\}} package). Dr. David Young at the Alabama Supercomputer Authority was instrumental in getting me set up on the HPC and I thank him for his patience with a persistent beginner, which helped ensure my timely graduation. I consider myself lucky to have been a member of the Ireland Center research group in the School of Fisheries, Aquaculture, and Aquatic Science: I would like to extend my gratitude to all past and present students in this group since 2014, who have provided me with tremendous support, comradery, and welcomed distractions from my work. Funding for this research was provided by the Arctic-Yukon-Kuskokwim Sustainable Salmon Initiative, and I am immensely grateful for their financial support. Finally, no acknowledgement would be complete without my loving wife, Michelle, for her ceaseless support and patience while I pursued my studies in quantitative fisheries science, otherwise known as ``fish math.''
\end{acknowledgments}

\newpage

\begin{singlespace}
	\tableofcontents
	\clearpage
	\listoffigures
	\clearpage
	\listoftables
\end{singlespace}

\end{romanpages}        % All done with roman-numbered pages

\normalem       % Make italics the default for \em

\setlength{\parskip}{0pt plus 0pt minus 0pt}

\doublespacing
