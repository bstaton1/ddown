\documentclass[12pt,]{book}
\usepackage{lmodern}
\usepackage{amssymb,amsmath}
\usepackage{ifxetex,ifluatex}
\usepackage{fixltx2e} % provides \textsubscript
\ifnum 0\ifxetex 1\fi\ifluatex 1\fi=0 % if pdftex
  \usepackage[T1]{fontenc}
  \usepackage[utf8]{inputenc}
\else % if luatex or xelatex
  \ifxetex
    \usepackage{mathspec}
  \else
    \usepackage{fontspec}
  \fi
  \defaultfontfeatures{Ligatures=TeX,Scale=MatchLowercase}
\fi
% use upquote if available, for straight quotes in verbatim environments
\IfFileExists{upquote.sty}{\usepackage{upquote}}{}
% use microtype if available
\IfFileExists{microtype.sty}{%
\usepackage{microtype}
\UseMicrotypeSet[protrusion]{basicmath} % disable protrusion for tt fonts
}{}
\usepackage[margin=1in]{geometry}
\usepackage{hyperref}
\hypersetup{unicode=true,
            pdftitle={Challenges and Tools in the Assessment and Management of Pacific Salmon Fisheries},
            pdfauthor={Ben Staton},
            pdfborder={0 0 0},
            breaklinks=true}
\urlstyle{same}  % don't use monospace font for urls
\usepackage{natbib}
\bibliographystyle{myapalike}
\usepackage{longtable,booktabs}
\usepackage{graphicx,grffile}
\makeatletter
\def\maxwidth{\ifdim\Gin@nat@width>\linewidth\linewidth\else\Gin@nat@width\fi}
\def\maxheight{\ifdim\Gin@nat@height>\textheight\textheight\else\Gin@nat@height\fi}
\makeatother
% Scale images if necessary, so that they will not overflow the page
% margins by default, and it is still possible to overwrite the defaults
% using explicit options in \includegraphics[width, height, ...]{}
\setkeys{Gin}{width=\maxwidth,height=\maxheight,keepaspectratio}
\IfFileExists{parskip.sty}{%
\usepackage{parskip}
}{% else
\setlength{\parindent}{0pt}
\setlength{\parskip}{6pt plus 2pt minus 1pt}
}
\setlength{\emergencystretch}{3em}  % prevent overfull lines
\providecommand{\tightlist}{%
  \setlength{\itemsep}{0pt}\setlength{\parskip}{0pt}}
\setcounter{secnumdepth}{5}
% Redefines (sub)paragraphs to behave more like sections
\ifx\paragraph\undefined\else
\let\oldparagraph\paragraph
\renewcommand{\paragraph}[1]{\oldparagraph{#1}\mbox{}}
\fi
\ifx\subparagraph\undefined\else
\let\oldsubparagraph\subparagraph
\renewcommand{\subparagraph}[1]{\oldsubparagraph{#1}\mbox{}}
\fi

%%% Use protect on footnotes to avoid problems with footnotes in titles
\let\rmarkdownfootnote\footnote%
\def\footnote{\protect\rmarkdownfootnote}

%%% Change title format to be more compact
\usepackage{titling}

% Create subtitle command for use in maketitle
\newcommand{\subtitle}[1]{
  \posttitle{
    \begin{center}\large#1\end{center}
    }
}

\setlength{\droptitle}{-2em}

  \title{Challenges and Tools in the Assessment and Management of Pacific Salmon
Fisheries}
    \pretitle{\vspace{\droptitle}\centering\huge}
  \posttitle{\par}
    \author{Ben Staton}
    \preauthor{\centering\large\emph}
  \postauthor{\par}
    \date{}
    \predate{}\postdate{}
  
\usepackage{booktabs}
%%% This is an example file for the Auburn University style options
%%%       aums.sty (Masters Thesis)
%%%       auphd.sty (Ph.D. Dissertation)
%%%       auhonors.sty (Honors Scholar)

%%%To use it, please edit the necessary options, title, author, date, year, keywords, advisor, professor, etc. 

% \documentclass[12pt]{report}
\usepackage{setspace}
% \usepackage{titlesec}
%\setcounter{secnumdepth}{3}
% \usepackage{aums}       % For Master's papers
\usepackage{auphd}     % For Ph.D.
%\usepackage{auhonors}  % For honors college
\usepackage[normalem]{ulem}       % underlining on style-page; see \normalem below
\usepackage{url}
\usepackage[table]{xcolor}
\usepackage{tikz}
\usepackage{pgf}
\usepackage{color,soul}
\usepackage{float}
\usepackage{caption}
\captionsetup{width=\textwidth}

\usepackage{amsmath,amsthm, amsfonts, mathrsfs, graphicx, setspace, fullpage, color}
\usepackage{natbib, appendix}
\usepackage[T1]{fontenc}
\usepackage{multirow}
\usepackage{mathabx}
\RequirePackage{adjustbox}
% \usepackage{epstopdf}
\AtBeginDocument{\renewcommand{\bibname}{References}}
\usepackage{hyperref}
%\usepackage{tocloft}
%\renewcommand\cftchapafterpnum{\vskip\baselineskip}
%\renewcommand\cftsecafterpnum{\vskip\baselineskip}
%\renewcommand\cftsubsecafterpnum{\vskip\baselineskip}
%\renewcommand\cftsubsubsecafterpnum{\vskip\baselineskip}
%\renewcommand\cftfigafterpnum{\vskip\baselineskip}
%\renewcommand\cfttabafterpnum{\vskip\baselineskip}

% remove double spacing from itemized lists
\usepackage{enumitem}
% \setlist[itemize]{noitemsep}
\setlist{before=\doublespacing,after=\doublespacing}

% citation style: remove the comma between author and year 
\setcitestyle{aysep={}}
% \setlength{\bibhang}{2em}

%%%%%Format rules: Normal margins are 1 in. If you need to print with 1.5in margins, uncomment the line below
%\oddsidemargin0.5in \textwidth6in

%% If you do not need a List of Abbreviations, then comment out the lines below and the \printnomenclature line.
%%for List of Abbreviations information:  (see http://www.mackichan.com/TECHTALK/509.htm  )
% \usepackage[intoc]{nomencl}
% \renewcommand{\nomname}{List of Abbreviations}   	       
% \makenomenclature 
%% don't forget to run:   makeindex ausample.nlo -s nomencl.ist -o ausample.nls
%% Also, if 

\makeatother
\let\oldmaketitle\maketitle
\AtBeginDocument{\let\maketitle\relax}

% Put the title, author, and date in. 
\title{Challenges and Tools in the Assessment and Management of Pacific Salmon Fisheries}
\author{Benjamin A. Staton} 
\date{May 5, 2019} %date of graduation
\copyrightyear{2019} %copyright year

\keywords{Fisheries management, Bayesian inference, decision analysis}

% Put the Thesis Adviser here. 
\adviser{Matthew J. Catalano}

% Put the committee here (including the adviser), one \professor for each. 
% The advisor must be first, and the dean of the graduate school must be last.
\professor{Matthew J. Catalano, \textit{PLEASE INDICATE YOUR AFFILIATION}}

\professor{Asheber Abebe, \textit{PLEASE INDICATE YOUR AFFILIATION}}

\professor{Lewis G. Coggins, Jr., \textit{PLEASE INDICATE YOUR AFFILIATION}}

\professor{Conor P. McGowan, \textit{PLEASE INDICATE YOUR AFFILIATION}}
\usepackage{booktabs}
\usepackage{longtable}
\usepackage{array}
\usepackage{multirow}
\usepackage[table]{xcolor}
\usepackage{wrapfig}
\usepackage{float}
\usepackage{colortbl}
\usepackage{pdflscape}
\usepackage{tabu}
\usepackage{threeparttable}
\usepackage{threeparttablex}
\usepackage[normalem]{ulem}
\usepackage{makecell}

\usepackage{amsthm}
\newtheorem{theorem}{Theorem}[chapter]
\newtheorem{lemma}{Lemma}[chapter]
\theoremstyle{definition}
\newtheorem{definition}{Definition}[chapter]
\newtheorem{corollary}{Corollary}[chapter]
\newtheorem{proposition}{Proposition}[chapter]
\theoremstyle{definition}
\newtheorem{example}{Example}[chapter]
\theoremstyle{definition}
\newtheorem{exercise}{Exercise}[chapter]
\theoremstyle{remark}
\newtheorem*{remark}{Remark}
\newtheorem*{solution}{Solution}
\begin{document}
\maketitle

\begin{romanpages}      % roman-numbered pages 

\TitlePage 

\doublespacing
\setlength{\parskip}{0pt plus 0pt minus 0pt}

\begin{abstract} 
\noindent
I'm going to write an abstract to go here. This paragraph will be a brief introduction chapter 1: the overall topic of the research

This is the second paragraph of the dissertation abstract, which will talk broadly about chapter 2: run timing forecast models.

This is the second paragraph of the dissertation abstract, which will talk broadly about chapter 3: in-season MSE models.

This is the third paragraph of the dissertation abstract, which will talk broadly about chapter 4: multi-stock population dynamics models and the best ways to inform management trade-offs. 

\end{abstract}

\begin{acknowledgments}
\noindent
Here is where I will thank everyone:

Catalano, Coggins, Connors, Jones, Dobson, Farmer, Fleischman, Smith, Liller, Esquible, Bechtol, Spaeder, Decossas, AL-HPC folks, Auburn Hopper HPC folks. Folks at the lab. Family and Michelle. RStudio staff. Instructors: Abebe (complex quant problems), Stevison (Shell/bash/HPC), McGowan (SDM), Sawant (GIS).  

\end{acknowledgments}

\begin{singlespace}
	\tableofcontents
	\clearpage
	\listoffigures
	\clearpage
	\listoftables
\end{singlespace}

% \printnomenclature[0.5in] %used for the List of Abbreviations
\end{romanpages}        % All done with roman-numbered pages

\normalem       % Make italics the default for \em

% \titlespacing\section{0pt}{12pt plus 4pt minus 2pt}{0pt plus 2pt minus 2pt}
% \titlespacing\subsection{0pt}{12pt plus 4pt minus 2pt}{0pt plus 2pt minus 2pt}
% \titlespacing\subsubsection{0pt}{12pt plus 4pt minus 2pt}{0pt plus 2pt minus 2pt}

\setlength{\parskip}{0pt plus 0pt minus 0pt}

\doublespacing

\chapter{Evaluation of Intra-Annual Harvest Control Rules For Kuskokwim
River Chinook salmon using Closed-Loop Simulation}\label{ch3}

\section{Introduction}\label{introduction}

\noindent
In-season harvest management of Pacific salmon fisheries in large river
systems is undertaken in the presence of a large amount of uncertainty
about how to schedule fishing opportunities. In order to manage in a
fully-informed way, a manager would require continuous and accurate
information on arrival timing, run size, fleet dynamics, and harvest.
With knowledge on these components, it would be theoretically possible
to perfectly harvest the surplus each year
\citep{adkison-cunningham-2015}. In reality, these quantities, when
estimates are available, are often highly uncertain
\citep{adkison-peterman-2000, flynn-hilborn-2004, hyun-etal-2012} which
results in difficult decision-making about how to best implement fishing
opportunities in order to meet a set of pre-defined objectives.

In addition to the substantial uncertainty in decision-making, there are
often sharp trade-offs between competing objectives, such as the desire
to provide adequate and equitable harvest opportunity versus the desire
to ensure adequate escapement \citep{catalano-jones-2014} and spreading
exploitation evenly among stock subcomponents
\citep{carney-adkison-2014, adkison-cunningham-2015}. When given the
task of balancing trade-offs such as these, the manager has the ability
to manipulate the fishing gear used as well as the spatiotemporal
distribution of fishing effort, though it is rarely clear as to how to
manipulate these management ``levers'' to achieve the desired outcomes.
Presumably, different strategies to performing these manipulations
(termed ``management strategies'' or ``harvest control rules'') will
exhibit differential performance at meeting the objectives and balancing
trade-offs, though without testing them it is difficult to have
confidence in which among them will provide the best chances of success.

Management Strategy Evaluation (MSE) has been proposed as a powerful
tool for determining how to manage exploited natural resource systems
with competing management objectives
\citep{cooke-1999, butterworth-2007}. MSE is a stochastic
simulation-based analytical technique whereby management strategies are
evaluated by comparing their relative performance at meeting pre-defined
objectives under simulated (though realistic) conditions. A management
strategy can be thought of as all of the steps that encompass the
collection of data, subsequent analyses, and resulting decision-making
surrounding the exploitation of a resource. The MSE approach tests a
range of such strategies to find the one(s) that are likely to be most
robust to uncertainty and balance trade-offs. This approach is powerful
as it can provide general insights without having to test strategies on
the real system, which would be incredibly time-intensive (each year is
one sample) and costly given that some candidate strategies can be risky
\citep{walters-martell-2004}. \citet{punt-etal-2014} outline a set of
seven steps to an MSE that must be conducted in order for the analysis
to be meaningful:

\begin{enumerate}
\def\labelenumi{(\arabic{enumi})}
\item
  identification of management objectives and performance measures for
  each; preferably under the direction of stakeholders and managers,
\item
  identification of the key uncertainties present in the system
  (biological, assessment, implementation, etc.),
\item
  identification of candidate management strategies for evaluation,
\item
  development of one or more models that serve as the representation of
  the real system including reasonably realistic representations of
  biological and fishery components (termed the ``operating model''),
\item
  selection of parameters to drive the operating model in accordance
  with the real system,
\item
  simulation of executing each strategy using the operating model(s),
  and
\item
  summary of performance measures, and presentation to managers and
  stakeholders.
\end{enumerate}

\noindent
While MSE analyses are most often used in multi-year evaluations
\citep{cooke-1999}, the same concept can be applied to evaluate the
performance of in-season strategies at meeting shorter-term objectives
as well.

Two broad classes of strategies could be conceived for in-season salmon
management: effort control using either a fixed schedule or a
more-involved (and data-intensive) process of opening and closing the
fishery based on in-season data \citep[i.e., management by emergency
order,][]{adkison-cunningham-2015}. There exist many substrategies that
fall into these two broad categories based on the characteristics of the
fishery and the timeliness and reliability of information available to
managers. \citet{carney-adkison-2014} evaluated these two strategies for
sockeye salmon stocks in Bristol Bay, Alaska, and found trade-offs
between maximizing harvest and reducing inter-annual variability in
harvest magnitude as well as spreading harvest pressure among substock
components. \citet{su-adkison-2002} evaluated a set of schedule-based
strategies that ranged in their aggressiveness and found differences in
strategy performance based on which objective carried most weight in
value functions, which of course implies that trade-offs exist.

An MSE analysis for subsistence salmon fisheries in large drainages
(such as the Yukon and Kuskokwim systems in Western Alaska) necessitates
different considerations than these two examples which focused on
commercial fisheries. While the types of strategies considered and
conservation-based objectives (adequate escapement and
temporally-distributed harvest) are broadly consistent, the fleet
dynamics and harvest-based objectives may be different. Subsistence
fishers are less concerned with maximizing harvest as they are with
maintaining consistent harvests that meet their needs between years and
that harvest opportunities allow exploitation consistent with cultural
practices (i.e., time of season and frequency of opportunities). The
fleet dynamics of subsistence fisheries are quite different than
commercial fisheries in that they are limited by processing capacity and
have a fixed targeted harvest for the season. Due to this processing
capacity, harvest of targeted species (such as Chinook salmon) in
subsistence fisheries are limited by the species composition, sometimes
expressed as a ratio of chum/sockeye:Chinook salmon. Subsistence fishers
must stop fishing when they reach their processing capacity, and when
this ratio is high (e.g., \textgreater{} 20), the catch will be
dominated by chum/sockeye. In-season harvest control rules have that
acknowledge these characteristics have not been evaluated for
subsistence salmon fisheries, highlighting a clear need for work that
focuses on this topic.

In this chapter, I investigate the performance of a variety of in-season
harvest control rules for subsistence salmon fisheries in large drainage
systems using a MSE approach. Though the analysis will be tailored to
the Kuskokwim River Chinook salmon subsistence fishery, the framework
developed will be general enough for application to other in-river
salmon fisheries in large drainages in which the primary users are
subsistence fishers. The objectives of the analysis will be to:

\begin{enumerate}
\def\labelenumi{(\arabic{enumi})}
\item
  develop a closed loop simulation model of the Kuskokwim River fishery
  system that allows simulation of a wide range of biological
  conditions,
\item
  assess the performance of several realistic harvest control rules that
  capture a range of complexity in their management dexterity and need
  for information, and
\item
  highlight the strength of trade-offs between competing objectives, and
  find control rules that might balance them better than others.
\end{enumerate}

\section{Methods}\label{methods}

I did some stuff.

\section{Results}\label{results}

I found some stuff.

\section{Discussion}\label{discussion}

Here's what it means.

\begin{singlespace}

\begin{table}[H]
\centering\rowcolors{2}{gray!6}{white}

\resizebox{\linewidth}{!}{
\begin{tabular}{c|>{\raggedright\arraybackslash}p{20em}>{\raggedright\arraybackslash}p{20em}cllcll}
\hiderowcolors
\toprule
\textbf{\#} & \textbf{Equation} & \textbf{Purpose/Description}\\
\midrule
\showrowcolors
1 & $N_s=N_{tot} \pi_s$ & Apportions total Chinook run size to subpopulations\\
2 & $p^{\prime}_{d,s} = \frac{e^{\frac{d-D_{50,s}}{h_s}}}{h_s \left(1 + e^{\frac{d-D_{50,s}}{h_s}} \right)^2}$ & Produces a time series of unstandardized entry timing values (logistic density function)\\
3 & $p_{d,s}=\frac{p^{\prime}_{d,s}}{\sum_d p^{\prime}_{d,s}}$ & Standardizes entry timing to sum to one for each Chinook subpopulation\\
4 & $A_{d,1,s}=N_s p_{d,s}$ & Populates first main stem reach with Chinook from each subpopulation\\
5 & $A_{d,1,4}=\phi_d \sum_{s=1}^3 A_{d,1,s}$ & Populates first reach with chum/sockeye main stem abundance\\
\addlinespace
6 & $S_{d,r,s}=\psi_{r,s} \cdot \left(A_{d,r,s} - H_{d,r,s} \right)$ & Generates escapement in each reach on each day from each population\\
7 & $A_{d+1,r+1,s}=A_{d,r,s}-H_{d,r,s}-S_{d,r,s}$ & Transition main stem survivors to the next reach on the next day\\
8 & $\text{logit}(p_{E,d,r})=\beta_0 + \beta_1 full_{d,r} + \beta_2 stop_{d,r} + \beta_3 \delta_{d-1,r,CH} + \beta_4 \delta_{d-1,r,CS} + \beta+5 \phi_{d,r}$ & Effort response model; $full$ and $stop$ are binary indicators; $\delta$ is the fraction of needed harvest obtained for Chinook ($CH$) and chum/sockeye ($CS$), and $\phi$ is the local species ratio\\
9 & $E_{d,r} p_{E,d,r} F_{d,r}$ & Generates realized effort in each reach on each day\\
10 & $H_{tot,d,r}=\text{min} \left(1 - e^{-E_{d,r} q} \sum_{s=1}^4 A_{d,r,s}, E_{d,r} F_{d,r} CPB_{max} \right)$ & Generates total salmon harvest by reach and day\\
\bottomrule
\end{tabular}}
\rowcolors{2}{white}{white}
\end{table}

\end{singlespace}

\emph{Insert Figures}

\setlength{\parskip}{6pt plus 2pt minus 1pt}

\bibliography{cites-without-doi.bib,cites-with-doi.bib}


\end{document}
