\documentclass[12pt,]{book}
\usepackage{lmodern}
\usepackage{amssymb,amsmath}
\usepackage{ifxetex,ifluatex}
\usepackage{fixltx2e} % provides \textsubscript
\ifnum 0\ifxetex 1\fi\ifluatex 1\fi=0 % if pdftex
  \usepackage[T1]{fontenc}
  \usepackage[utf8]{inputenc}
\else % if luatex or xelatex
  \ifxetex
    \usepackage{mathspec}
  \else
    \usepackage{fontspec}
  \fi
  \defaultfontfeatures{Ligatures=TeX,Scale=MatchLowercase}
\fi
% use upquote if available, for straight quotes in verbatim environments
\IfFileExists{upquote.sty}{\usepackage{upquote}}{}
% use microtype if available
\IfFileExists{microtype.sty}{%
\usepackage{microtype}
\UseMicrotypeSet[protrusion]{basicmath} % disable protrusion for tt fonts
}{}
\usepackage[margin=1in]{geometry}
\usepackage{hyperref}
\hypersetup{unicode=true,
            pdftitle={Challenges and Tools in the Assessment and Management of Pacific Salmon Fisheries},
            pdfauthor={Ben Staton},
            pdfborder={0 0 0},
            breaklinks=true}
\urlstyle{same}  % don't use monospace font for urls
\usepackage{natbib}
\bibliographystyle{apalike}
\usepackage{longtable,booktabs}
\usepackage{graphicx,grffile}
\makeatletter
\def\maxwidth{\ifdim\Gin@nat@width>\linewidth\linewidth\else\Gin@nat@width\fi}
\def\maxheight{\ifdim\Gin@nat@height>\textheight\textheight\else\Gin@nat@height\fi}
\makeatother
% Scale images if necessary, so that they will not overflow the page
% margins by default, and it is still possible to overwrite the defaults
% using explicit options in \includegraphics[width, height, ...]{}
\setkeys{Gin}{width=\maxwidth,height=\maxheight,keepaspectratio}
\IfFileExists{parskip.sty}{%
\usepackage{parskip}
}{% else
\setlength{\parindent}{0pt}
\setlength{\parskip}{6pt plus 2pt minus 1pt}
}
\setlength{\emergencystretch}{3em}  % prevent overfull lines
\providecommand{\tightlist}{%
  \setlength{\itemsep}{0pt}\setlength{\parskip}{0pt}}
\setcounter{secnumdepth}{5}
% Redefines (sub)paragraphs to behave more like sections
\ifx\paragraph\undefined\else
\let\oldparagraph\paragraph
\renewcommand{\paragraph}[1]{\oldparagraph{#1}\mbox{}}
\fi
\ifx\subparagraph\undefined\else
\let\oldsubparagraph\subparagraph
\renewcommand{\subparagraph}[1]{\oldsubparagraph{#1}\mbox{}}
\fi

%%% Use protect on footnotes to avoid problems with footnotes in titles
\let\rmarkdownfootnote\footnote%
\def\footnote{\protect\rmarkdownfootnote}

%%% Change title format to be more compact
\usepackage{titling}

% Create subtitle command for use in maketitle
\newcommand{\subtitle}[1]{
  \posttitle{
    \begin{center}\large#1\end{center}
    }
}

\setlength{\droptitle}{-2em}

  \title{Challenges and Tools in the Assessment and Management of Pacific Salmon
Fisheries}
    \pretitle{\vspace{\droptitle}\centering\huge}
  \posttitle{\par}
    \author{Ben Staton}
    \preauthor{\centering\large\emph}
  \postauthor{\par}
    \date{}
    \predate{}\postdate{}
  
\usepackage{booktabs}
%%% This is an example file for the Auburn University style options
%%%       aums.sty (Masters Thesis)
%%%       auphd.sty (Ph.D. Dissertation)
%%%       auhonors.sty (Honors Scholar)

%%%To use it, please edit the necessary options, title, author, date, year, keywords, advisor, professor, etc. 

% \documentclass[12pt]{report}
\usepackage{setspace}
% \usepackage{titlesec}
%\setcounter{secnumdepth}{3}
% \usepackage{aums}       % For Master's papers
\usepackage{auphd}     % For Ph.D.
%\usepackage{auhonors}  % For honors college
\usepackage[normalem]{ulem}       % underlining on style-page; see \normalem below
\usepackage{url}
\usepackage[table]{xcolor}
\usepackage{tikz}
\usepackage{pgf}

\usepackage{amsmath,amsthm, amsfonts, mathrsfs, graphicx, setspace, fullpage, color}
\usepackage{natbib, appendix}
\usepackage[T1]{fontenc}
\usepackage{multirow}
\usepackage{mathabx}
\RequirePackage{adjustbox}
% \usepackage{epstopdf}
\AtBeginDocument{\renewcommand{\bibname}{References}}
\usepackage{hyperref}
%\usepackage{tocloft}
%\renewcommand\cftchapafterpnum{\vskip\baselineskip}
%\renewcommand\cftsecafterpnum{\vskip\baselineskip}
%\renewcommand\cftsubsecafterpnum{\vskip\baselineskip}
%\renewcommand\cftsubsubsecafterpnum{\vskip\baselineskip}
%\renewcommand\cftfigafterpnum{\vskip\baselineskip}
%\renewcommand\cfttabafterpnum{\vskip\baselineskip}

% remove double spacing from itemized lists
\usepackage{enumitem}
% \setlist[itemize]{noitemsep}
\setlist{before=\singlespacing,after=\singlespacing}


%%%%%Format rules: Normal margins are 1 in. If you need to print with 1.5in margins, uncomment the line below
%\oddsidemargin0.5in \textwidth6in

%% If you do not need a List of Abbreviations, then comment out the lines below and the \printnomenclature line.
%%for List of Abbreviations information:  (see http://www.mackichan.com/TECHTALK/509.htm  )
% \usepackage[intoc]{nomencl}
% \renewcommand{\nomname}{List of Abbreviations}   	       
% \makenomenclature 
%% don't forget to run:   makeindex ausample.nlo -s nomencl.ist -o ausample.nls
%% Also, if 

\makeatother
\let\oldmaketitle\maketitle
\AtBeginDocument{\let\maketitle\relax}

% Put the title, author, and date in. 
\title{Challenges and Tools in the Assessment and Management of Pacific Salmon Fisheries}
\author{Benjamin A. Staton} 
\date{May 5, 2019} %date of graduation
\copyrightyear{2019} %copyright year

\keywords{Fisheries management, Bayesian inference, decision analysis}

% Put the Thesis Adviser here. 
\adviser{Matthew J. Catalano}

% Put the committee here (including the adviser), one \professor for each. 
% The advisor must be first, and the dean of the graduate school must be last.
\professor{Matthew J. Catalano, AFFILIATION}

\professor{Asheber Abebe, AFFILIATION}

\professor{Lewis G. Coggins, Jr., AFFILIATION}

\professor{Conor P. McGowan, AFFILIATION}
\usepackage{booktabs}
\usepackage{longtable}
\usepackage{array}
\usepackage{multirow}
\usepackage[table]{xcolor}
\usepackage{wrapfig}
\usepackage{float}
\usepackage{colortbl}
\usepackage{pdflscape}
\usepackage{tabu}
\usepackage{threeparttable}
\usepackage{threeparttablex}
\usepackage[normalem]{ulem}
\usepackage{makecell}

\usepackage{amsthm}
\newtheorem{theorem}{Theorem}[chapter]
\newtheorem{lemma}{Lemma}[chapter]
\theoremstyle{definition}
\newtheorem{definition}{Definition}[chapter]
\newtheorem{corollary}{Corollary}[chapter]
\newtheorem{proposition}{Proposition}[chapter]
\theoremstyle{definition}
\newtheorem{example}{Example}[chapter]
\theoremstyle{definition}
\newtheorem{exercise}{Exercise}[chapter]
\theoremstyle{remark}
\newtheorem*{remark}{Remark}
\newtheorem*{solution}{Solution}
\begin{document}
\maketitle

\begin{romanpages}      % roman-numbered pages 

\TitlePage 

\doublespacing
\setlength{\parskip}{0pt plus 0pt minus 0pt}

\begin{abstract} 
\noindent
I'm going to write an abstract to go here. This is the first paragraph of the dissertation abstract, which will talk about chapter 1..

This is the second paragraph of the dissertation abstract, which will talk broadly about chapter 2.

This is the third paragraph of the dissertation abstract, which will talk broadly about chapter 3.

This is the fourth paragraph of the dissertation abstract, which will talk broadly about chapter 4.
\end{abstract}

\begin{acknowledgments}
\noindent
Here is where I will thank everyone.

Matt, Lew, Brendan, Mike, Sam, AL-HPC folks, Steve, Nick, Zach, Janessa. Family and Michelle. Folks at the lab. RStudio staff.
\end{acknowledgments}

\begin{singlespace}
	\tableofcontents
	\clearpage
	\listoffigures
	\clearpage
	\listoftables
\end{singlespace}

% \printnomenclature[0.5in] %used for the List of Abbreviations
\end{romanpages}        % All done with roman-numbered pages

\normalem       % Make italics the default for \em

% \titlespacing\section{0pt}{12pt plus 4pt minus 2pt}{0pt plus 2pt minus 2pt}
% \titlespacing\subsection{0pt}{12pt plus 4pt minus 2pt}{0pt plus 2pt minus 2pt}
% \titlespacing\subsubsection{0pt}{12pt plus 4pt minus 2pt}{0pt plus 2pt minus 2pt}

\setlength{\parskip}{0pt plus 0pt minus 0pt}

\doublespacing

\chapter*{Preface}\label{preface}
\addcontentsline{toc}{chapter}{Preface}

I used bookdown \citep{xie-2015} to make this document.

\chapter{Simulation-based Evaluation of Assessment Approaches for
Single-Species, Mixed-stock Pacific Salmon Fisheries}\label{ch4}

\section{Introduction}\label{introduction}

\noindent
Many salmon populations in large drainage systems are harvested
primarily in a relatively small spatial area and are managed as a single
stock (i.e., the concept of a ``mixed-stock fishery''). However, these
``stocks'' are instead stock-complexes, in which the aggregate stock is
comprised of several (and sometimes, many) substocks. These substocks
are known to show differences in genotypic (Templin et al. 2004),
phenotypic (e.g., morphology; Hendry and Quinn 1997), behavioral (e.g.,
run timing; Clark et al. 2015, Smith and Liller 2017), and life history
(i.e., age-at-maturation, Blair et al. 1993) characteristics that are
the result of adaptations to local environments. It has been widely
proposed that maintaining this diversity of local adaptation (hereafter,
``biodiversity'') is favorable both from ecosystem and exploitation
perspectives (i.e., the statistical dampening of random variability in a
system made up of many additive random processes, otherwise known as the
``portfolio effect''; Schindler et al. 2010, Schindler et al. 2015).
This level of variability in substock-specific characteristics can
ultimately lead to heterogeneity in productivity among the substock
components (Walters and Martell 2004). Productivity is the ability of
the substock to replace itself after harvesting, often represented for
salmon populations as the maximum number of recruits (future migrating
adults before harvest) per one spawner, which (due to density-dependent
survival) is attained at very low numbers of spawners (hereafter,
\(\alpha\)). Stocks with higher \(\alpha\) values can sustain greater
exploitation rates than stocks with smaller values, in fact, \(\alpha\)
can be expressed in terms of the exploitation rate that maximizes
sustained yield (Schnute and Kronlund 2002):

\begin{equation}
  \alpha=\frac{e^{U_{\text{MSY}}}}{1 - U_{\text{MSY}}}.
  \label{eq:umsy-to-alpha}
\end{equation}

Given that there is likely some level heterogeneity in \(\alpha_j\) and
\(U_{\text{MSY},j}\) among individual substocks \(j\), the logical
conclusion is that in a mixed-stock fishery where \(U_t\) is common
among all substocks, some weaker substocks must be exploited at
\(U_t > U_{\text{MSY},j}\) in order to fish the more productive
substocks at \(U_{\text{MSY},j}\). This of course implies a trade-off,
and in some cases it might be necessary to over-exploit some substocks
in order to maximize harvest (Figure \ref{fig:trade-off-plot}, Walters
and Martell 2004).

Before these trade-offs are considered by managers in a well-informed
way, the shape and magnitude of the trade-off must first be quantified
as shown in Figure \ref{fig:trade-off-plot}. Figures like this are
generated using the estimated productivity and carrying capacity of all
(or a representative sample) of the substocks within a mixed-stock
fishery. These quantities are obtained using a spawner-recruit analysis,
which involves tracking the number of recruits that were produced in
each brood year (i.e., parent year) by the number of fish that spawned
in the same calendar year and fitting a curve to the resulting pattern.
The spawner-recruit literature is extensive, but primarily focuses
primarily on assessing single populations as opposed to substock
components (but see the work of Skeena River sockeye substocks Walters
et al. 2008; Korman and English 2013). In my mind, this is due to two
factors:

\begin{enumerate}
\def\labelenumi{(\arabic{enumi})}
\item
  the data to perform well-informed substock-specific spawner-recruit
  analyses are often unavailable (20-30 years of continuous spawner and
  harvest counts/estimates and age composition for each substock) and
\item
  management actions are often not precise enough to target particular
  substocks in the fishery, so deriving substock-specific estimates
  could be of little utility.
\end{enumerate}

\noindent
This proposed chapter will pertain to salmon systems for which there is
a reasonable amount of data available for a significant portion of the
substocks and in situations where spawner-recruitment analysis estimates
are desired for each.

The methods to fit spawner-recruit models can be grouped into two broad
categories: time-independent error models (i.e., Clark et al. 2009) and
state-space (i.e., time series) models (Fleischman et al. 2013, Su and
Peterman 2012). The independent error models typically take on a
regression analytical method, and is thus subject to substantial
pitfalls (Walters and Martell 2004). The state-space class of models
captures the process of recruitment events leading to future spawners
while simultaneously accounting for variability in the biological and
measurement processes that gave rise to the observed data (de Valpine
and Hastings 2002, Fleischman et al. 2013). Including this level of
additional model complexity comes at computational costs, as these
models are best-suited for Bayesian inference with Markov Chain Monte
Carlo (MCMC) methods, but has been shown to reduce bias in estimates in
some circumstances (Su and Peterman 2012, Walters and Martell 2004).

There has been recent interest in using multi-stock state-space spawner
recruit models for policy analyses that incorporate notions of substock
diversity as well as other fishery objectives (e.g., temporal stability
of harvests). Before strong inferences can be made from such analyses,
the performance of the estimation models used to parameterize them needs
to be evaluated, as well as the appropriate level of model complexity.
In this final chapter, I will evaluate the performance of a range of
assessment models for mixed-stock salmon fisheries via
simulation-estimation. The objectives will be to

\begin{enumerate}
\def\labelenumi{(\arabic{enumi})}
\item
  develop a set of varyingly-complex multi-stock versions of the
  state-space spawner-recruit models that have been rapidly gaining
  popularity, particularly in Alaska (Walters and Martell 2004, Su and
  Peterman 2012, Fleischman et al. 2013, Staton et al. 2017),
\item
  determine the sensitivity of trade-off conclusions to assessment model
  complexity using empirical data from Kuskokwim River Chinook salmon
  substocks, and
\item
  test the performance of the assessment models \emph{via}
  simulation-estimation.
\end{enumerate}

\section{Methods}\label{methods}

This analysis will be conducted in both an empirical and a
simulation-estimation framework to evaluate the sensitivity and
performance of assessment strategies for the mixed-stock assessment
problem in Pacific salmon fisheries. First, all assessment methods will
be fitted to observed data from the Kuskokwim River substocks (\(n_j\) =
13) for the empirical objective. Then, a hypothetical system will be
generated with known dynamics and will be comprised of several
age-structured substocks. Then, these hypothetical populations will be
sampled per a realistic sampling scheme (i.e., frequency of sampling,
appropriate levels of observation variance, etc.). Each of the
assessment models will be fitted to the resulting data sets, and the
management quantities \(U_{\text{MSY}}\) and \(S_{\text{MSY}}\) (both on
an aggregate and substock basis) will be calculated from the resulting
estimates. The estimated quantities will then be compared to the true
driving parameters and will be summarized and model performance will be
compared among a set of competing estimation models. Inference from the
simulation regarding which assessment models perform the best can then
be used to justify an appropriate level of model complexity for this
problem. I will begin by describing the estimation models assessed in
this study and then provide details on the empirical and simulation
analyses.

\subsection{Regression-based models}\label{regression-based-models}

Two regression-based approaches to estimating Ricker (1954)
spawner-recruit parameters in the multi-stock case were assessed: (a) a
single mixed-effect regression model with random intercepts and (b)
\(n_j\) independent regression models. A description and justification
of each method is provided in the sections that follow.

\subsubsection{Mixed-effect linear
regression}\label{mixed-effect-linear-regression}

The Ricker (1954) spawner-recruit model can be written as:

\begin{equation}
  R_y=\alpha S_y e^{-\beta S_y + \varepsilon_y}
  \label{eq:basic-ricker}
\end{equation}

\noindent
where \(R_y\) is the total recruitment produced by the escapement
\(S_y\) in brood year \(y\), \(\alpha\) is the maximum
recruits-per-spawner (RPS), \(\beta\) is the inverse of the escapement
that produces maximum recruitment (\(S_{\text{MAX}}\)), and
\(\varepsilon_y\) are independent mean zero normal random variables
attributed solely to environmental fluctuations. Primary interest lies
in estimating the population dynamics parameters \(\alpha\) and
\(\beta\) as they can be used to obtain biological reference points off
of which sustainable policies can be developed. This function is
increasing at small escapements and declining at large ones, though can
be linearized:

\begin{equation}
  \log(\text{RPS}_y)=\log(\alpha)-\beta S_y + \varepsilon_y,
  \label{eq:lin-ricker-fixed}
\end{equation}

\noindent
allowing for estimation of the parameters log(\(\alpha\)) and \(\beta\)
in a linear regression framework using the least squares method (Clark
et al. 2009). This relationship is nearly always declining, implying a
compensatory effect on survival (i.e., RPS) with reductions in spawner
abundance (Rose et al. 2002). Regression-based methods to estimating
spawner-recruit parameters are well known to be fraught with two primary
issues: (1) ignoring the intrinsic time linkage whereby brood year
recruits (part of the response variable) make up the escapement for the
one or more future brood years (the predictor variable), which then
produce the future recruits (response variables) and (2) ignoring the
fact that escapement and harvest are often measured with substantial
error. The first issue is known as the ``time-series bias'', and is
known to chronically cause positive biases in \(\alpha\) and negative
biases in \(\beta\), causing the same directional biases in
\(U_{\text{MSY}}\) and \(S_{\text{MSY}}\), respectively (i.e.,
spuriously providing too aggressive harvest policy recommendations;
Walters 1985). The second is known as the ``errors-in-variables bias''
and is known to cause an apparent (but false) scatter which inserts
variability that commonly-used regression estimators do not account for
(Walters and Ludwig 1981). Though these methods have been known for
their problems for over 30 years, they are still somewhat widely used
(Korman and English 2013).

It is not difficult to conceive a multi-stock formulation of this model
by including substock-specific random effects on the intercept
{[}log(\(\alpha\)){]}:

\begin{equation}
  \log(\text{RPS}_{y,j})=\log(\alpha_j)-\beta_j S_y,j + \varepsilon_y,
  \label{eq:lin-ricker-mixed}
\end{equation}

\noindent
where

\begin{equation}
  \log(\alpha_j)=\log(\alpha) + \varepsilon_{\alpha,j},
  \label{eq:random-alpha}
\end{equation}

\noindent
and

\begin{equation}
  \varepsilon_{\alpha,j} \sim \text{N}(0,\sigma_{\alpha}).
  \label{eq:random-alpha-errors}
\end{equation}

It does not make sense to include stock-level random effects on the
slope, given that \(\beta\) is a capacity parameter related to the
compensatory effect of resource limitation experienced by juveniles,
likely in the freshwater environment (i.e., amount of habitat as opposed
to quality of habitat). Fitting the individual substock models in this
hierarchical fashion allows for the sharing of information such that the
more intensively-assessed substocks can help inform those that are more
data-poor.

\subsubsection{Independent regression
models}\label{independent-regression-models}

\noindent
The mixed-effect model may have the benefit of sharing information to
make some substocks more estimable, but it should also have the tendency
to pull the extreme \(\alpha_j\) (those in the tails of the
hyperdistribution) toward \(\alpha\). This behavior may not be
preferable for policy recommendations, as it should tend to dampen the
extent of heterogeneity estimated in \(\alpha_j\). For this reason,
independent regression estimates for each substock will also be obtained
(i.e., the full fixed effects model) for evaluation.

\subsubsection{Brood table
reconstruction}\label{brood-table-reconstruction}

An important point in the use of the regression-based method is in how
\(\text{RPS}_{y,j}\) is obtained. Only \(S_{y,j}\) is directly observed;
\(R_{y,j}\) is observed (for Chinook salmon) over four calendar years as
not all fish mature and make the spawning migration at the same age.
Thus, in order to completely observe one \(\text{RPS}_y\) outcome,
escapement must be monitored in year \(y\) and escapement, harvest, and
age composition must be monitored in the subsequent years \(y + 4\),
\(y + 5\), \(y + 6\) and \(y + 7\). Thus, it is easy to see how missing
one year of sampling (which is an incredibly common occurrence, Figure
\ref{fig:obs-freq}) can lead to issues with this approach. Only
completely observed \(\text{RPS}_{y,j}\) observations will be used for
this analysis, with the exception of missing age count data. For
substocks with no age composition data, the average age composition
across substocks that have data will be used to reconstruct
\(\text{RPS}_{y,j}\), but will be provided only for years with
escapement sampling for substock \(j\). Only substocks with \(\ge3\)
completely observed pairs of \(S_{y,j}\) and \(\text{RPS}_{y,j}\) were
fitted.

\subsection{The full state-space
model}\label{the-full-state-space-model}

\noindent
There will be four versions of the state-space formulation. As three
versions are simplifications of the full model, the full model will be
presented completely and the changes resulting in the other three model
structures will be described in the subsequent section. The state-space
formulation of a multi-stock spawner recruit analysis developed and
evaluated here is an extension of various single-stock versions (e.g.,
Fleischman et al. 2013). Walters et al. (2008) used a similar model
using maximum likelihood methods to provide estimates of
\textgreater{}50 substocks in the Skeena River drainage, British
Columbia. The model presented here will be fitted in the Bayesian mode
of inference using the program JAGS (Plummer 2017), and will relax
certain assumptions made by Walters et al. (2008) such as the important
notion of perfectly-shared recruitment residuals (i.e., anomalies --
deviations from the expected population response). It will also have the
ability to relax the assumption of constant maturity schedules across
brood years. See Table \ref{tab:ch4-notation-table} for a description of
the index notation, in particular note the difference between the brood
year index \(y\) and the calendar year index \(t\).

The state-space model can be partitioned into two submodels: (a) the
process submodel which generates the true states of \(R_{y,j}\) and the
resulting calendar year states (e.g., \(S_{t,j}\)) and (b) the
observation submodel which fits the observed data to the true states.
The model is fitted to three primary data sources:

\begin{enumerate}
\def\labelenumi{(\arabic{enumi})}
\item
  escapement counts from the \(n_j\) substocks with data observed over
  \(n_t\) calendar years (some of which may be missing observations),
\item
  \(n_t\) calendar year estimates of aggregate harvest summed across all
  substocks included in the analysis, and
\item
  vectors of length \(n_a\) representing the calendar year age
  composition (relative contribution of each age class to the total run)
  for all substocks that have this information.
\end{enumerate}

\noindent
Note that this method allows for missing calendar year observations and
does not require excluding brood year recruitment events that are not
fully observed as was done for the regression-based models.

\subsubsection{Process submodel: brood year
processes}\label{process-submodel-brood-year-processes}

The recruitment process operates by producing a mean prediction from a
deterministic Ricker (1954) relationship (Equation
\eqref{eq:basic-ricker}) for \(n_y\) brood years for each of the \(n_j\)
substocks. From these deterministic predictions, auto-correlated process
variability is added to generate the true realized state. To populate
the first \(n_a\) calendar year true states with recruits of each age
\(a\), the first \(a_{max}\) brood year expected recruitment states are
not linked to a spawner abundance through Equation
\eqref{eq:basic-ricker}, but instead will be assumed to have a constant
mean equal to unfished equilibrium recruitment (where non-zero \(S_j\)
produces \(R_j = S_j\) when unexploited and in the absence of process
variability):

\begin{equation}
  \bar{R}_{y,j}=\frac{\log(\alpha_j)}{\beta_j},
  \label{eq:unfished-R0}
\end{equation}

\noindent
where \(\bar{R}_{y,j}\) is the expected (i.e., deterministic)
recruitment in brood year \(y\) from substock \(j\) with Ricker
parameters \(\alpha_j\) and \(\beta_j\). The remaining \(n_y - a_{max}\)
brood years will have an explicit time linkage:

\begin{equation}
  \bar{R}_{y,j} = \alpha_j S_{t,j} e^{-\beta_j S_{t,j}},
  \label{eq:tsm-ricker-pred}
\end{equation}

\noindent
where \(t = y-a_{max}\) is the \(t^{\text{th}}\) calendar year index in
which the escapement produced the recruits in the \(y^{\text{th}}\)
brood year index.

From these deterministic predictions of the biological recruitment
process, auto-correlated lag-1 process errors will be added to produce
the true realized states:

\begin{equation}
  \log(R_{y,1:n_j}) \sim \text{MVN}\left(\log(\bar{R}_{y,1:n_j}) + \omega_{y,1:n_j}, \Sigma_R\right),
  \label{eq:tsm-ricker-anomalies}
\end{equation}

\noindent
where

\begin{equation}
  \omega_{y,1:n_j} = \phi \left(\log(R_{y-1,j}) - \log(\bar{R}_{y-1,j}) \right),
  \label{eq:tsm-omega}
\end{equation}

\noindent
where \(R_{y,1:n_j}\) is a vector of true recruitment states across the
\(n_j\) stocks in brood year \(y\), \(\omega_{y,1:n_j}\) is the portion
of the total process error attributable to serial auto-correlation,
\(\phi\) is the lag-1 auto-correlation coefficient, and \(\Sigma_R\) is
a covariance matrix representing the white noise portion of the total
recruitment process variance. The covariance matrix \(\Sigma_R\) will be
estimated such that each substock will have a unique variance and
covariance with each other substock. The multivariate normal errors are
on the log scale so that the variability on \(R_{y,j}\) is lognormal,
which is the most commonly used error distribution for describing
spawner-recruit data sets (Walters and Martell 2004). Further, the
multivariate normal will be used as opposed to \(n_j\) separate normal
distributions so that the degree of synchrony in brood-year recruitment
deviations (i.e., process errors) among substocks is captured and freely
estimated.

The maturity schedule is an important component of age-structured
spawner-recruit models, as it determines which calendar years the brood
year recruits \(R_{y,j}\) return to spawn (and be observed). Recent
state-space spawner-recruit analyses have accounted for brood year
variability in maturity schedules as Dirichlet random vectors drawn from
a common hyperdistribution characterized by a mean maturation-at-age
probability vector (\(\pi_{1:n_a}\)) and an inverse dispersion parameter
(\(D\)) (see Fleischman et al. 2013, Staton et al. 2017 for
implementation in JAGS), and the same approach will be used here with
maturity schedules shared perfectly among substocks within a brood year.
Brood year-specific maturity schedules will be treated as random
variables such that:

\begin{equation}
  p_{y,a} \stackrel{\text{iid}}{\sim} \text{Dir}(\pi_{1:n_a} D). 
  \label{eq:dirichlet}
\end{equation}

\noindent
where \(p_{y,a}\) is the probability a fish spawned in brood year \(y\)
will mature at age \(a\). While there is almost certainly some level of
between-substock variability in average maturity schedules, I have made
many attempts to estimate it and include it in the model, but all
efforts resulted in either (1) nonsensical maturity estimates, (2)
systematic residual patterns among substocks with and without age
composition data, or (3) require auxiliary (i.e., never observed)
information for substocks that do not have age composition information
in order to fit. This result indicates the variability is not estimable
from the available data. Additionally, I think it is reasonable to
expect brood year deviations should be similar between substocks given
that the factors that set the probability of maturing at age are likely
linked to growth and mortality conditions in the ocean part of the
life-cycle, in which case all substocks would experience similar
conditions.

\subsubsection{Process submodel: calendar-year
processes}\label{process-submodel-calendar-year-processes}

\noindent
In order to link \(R_{y,j}\) with calendar year observations of
escapement from each substock, the \(R_{y,j}\) will be allocated to
calendar year runs:

\begin{equation}
  N_{t,j}=\sum_{a=1}^{n_a} R_{t+n_a-a,j} p_{t+n_a-a,a},
  \label{eq:tsm-get-N}
\end{equation}

\noindent
where \(N_{t,j}\) is the run abundance in calendar year \(t\) from
substock \(j\). The harvest process will be modeled using a freely
estimated annual exploitation rate (\(U_t\)) time series for
fully-vulnerable substocks:

\begin{equation}
  H_{t,j}=N_{t,j} U_t v_j,
  \label{eq:tsm-get-H}
\end{equation}

\noindent
and escapement will be obtained as:

\begin{equation}
  S_{t,j}=N_{t,j} (1 - U_t v_j),
  \label{eq:tsm-get-S}
\end{equation}

\noindent
where \(v_j\) are substock-specific vulnerabilities to harvest (1 =
fully vulnerable; 0 = not vulnerable at all). For the analysis of
empirical Kuskokwim River data, these quantities will be externally
reconstructed by region using historical run and harvest timing. For the
simulation analysis, all substocks will be assumed fully vulnerable for
simplification. The quantities \(N_t\) and \(S_t\) aggregated among all
substocks can be obtained by summing within a \(t\) index across the
\(j\) indices. Calendar year age composition for each substock will be
obtained by dividing an age-structured vector of the aggregate run at
year \(t\) and age \(a\) by the total aggregate run in year \(t\).

\subsubsection{Observation submodel}\label{observation-submodel}

\noindent
Three data sources will be used to fit the model: observed (estimated)
escapement from each substock (\(S_{obs,t,j}\)) with assumed known
coefficients of variation (CV), total harvest arising from the aggregate
stock (\(H_{obs,t}\)) with assumed known CV, and the age composition of
substocks with age composition (the substocks monitored using weirs;
\(n = 6\) for the Kuskokwim River) each calendar year
(\(q_{obs,t,a,j}\)) (which has associated effective sample size
\(ESS_{t,j}\) equal to the number of fish successfully aged for substock
\(j\) in year \(t\)). The CVs will be converted to lognormal standard
deviations:

\begin{equation}
  \sigma_{\text{log}}=\sqrt{\log(\text{CV}^2+1)},
  \label{eq:cv2sig}
\end{equation}

\noindent
and used in lognormal likelihoods to fit the time series \(S_{t,j}\) to
\(S_{obs,t,j}\) and \(H_t\) to \(H_{obs,t}\). Calendar year age
composition will be fitted using parameter vectors \(q_{t,1:n_a,j}\) and
observed vectors of (\(q_{obs,t,1:n_a,j} ESS_{t,j}\)).

\subsection{Alternate state-space
models}\label{alternate-state-space-models}

Three alternate formulations of the state-space model will be evaluated,
and all are simplifications of the full model described above regarding
the structure of (1) the covariance matrix on recruitment residuals and
(2) the maturity process. The simplest model will not include brood year
variability in maturity schedules and \(\Sigma_R\) will be constructed
by estimating a single \(\sigma_R^2\) and \(\rho\), and populating the
diagonal elements with \(\sigma_R^2\) and off-diagonal elements with
\(\rho \sigma_R^2\). One drawback of constructing \(\Sigma_R\) this way
is that \(\rho < -0.05\) for a \(13 \times 13\) covariance matrix
results in positive-indefiniteness, which is prohibited by JAGS. Thus, a
constraint is required to maintain \(-0.05 \le \rho < 1\) to prevent the
sampler from crashing. In one intermediate model, brood year maturation
variability will be included but the covariance matrix will be
constructed as in the simplest model. In the other intermediate model,
brood year variability in maturation will not be included but the
covariance matrix will be fully estimated as in the full model.

\subsection{Analysis of Kuskokwim River substock
data}\label{analysis-of-kuskokwim-river-substock-data}

\subsubsection{Data sources}\label{data-sources}

AYKDBMS

\subsubsection{Data preparation}\label{data-preparation}

\emph{Need to turn partial aerial surveys into total annual escapement
estimates}

\paragraph{Substock escapement}\label{substock-escapement}

\noindent
For substocks monitored \emph{via} weir, \(S_{obs,t,j}\) was taken to be
the total estimated weir passage each year (\textbf{CITE ADF\&G REPORT})
with a CV of 5\%. Substocks monitored \emph{via} aerial survey needed
special care, however. Surveys have been flown only once per year on a
relatively small fraction of each tributary system, resulting in them
being indices of escapement rather than estimates of total escapement.
The later of these two information sources was desired however, because
it allows calculation of biological reference points that are expressed
in terms of the scale of the population (e.g., \(S_{\text{MAX}}\)),
rather than as a rate (i.e., \(U_t\)). Note that if only estimates of
\(U_{\text{MSY}}\) were required, no accounting for the partial count
would be necessary.

The approach developed to estimate total escapement from single-pass
aerial surveys involved:

\begin{enumerate}
\def\labelenumi{(\alph{enumi})}
\item
  mapping the distribution of detected telemetry-tagged Chinook salmon
  against distribution of the aerial survey counts. This comparison
  allowed for an expansion to estimate how many salmon would have been
  counted if the entire tributary had been flown.
\item
  obtaining and applying an ``observability'' correction factor for the
  temporal problem of counting a dynamic pool at one point in its
  trajectory. This correction factor was based on the relationship
  between paired weir and aerial counts on \(n=3\) of the systems in the
  analysis.
\end{enumerate}

\noindent
\emph{Spatial expansion}

\noindent
The core of the the spatial expansion estimator is the assumption:

\begin{equation}
  \frac{A_{f,t,j}}{T_{f,t,j}} = \frac{A_{u,t,j}}{T_{u,t,j}},
  \label{eq:air-expand1}
\end{equation}

\noindent
where the quantities \(A\) and \(T\) represent fish and tags,
respectively in flown (\(A_f\) and \(T_f\)) and unflown (\(A_u\) and
\(T_u\)) reaches in year \(t\) and for substock \(j\). This assumption
states that the ratio of spawners per one tagged spawner is the same
between flown and unflown river sections at the time of the aerial index
count and the aerial telemetry flights. Equation \eqref{eq:air-expand1}
and can be rearranged as:

\begin{equation}
  A_{u,t,j} = A_{f,t,j} \frac{T_{u,t,j}}{T_{f,t,j}},
  \label{eq:air-expand2}
\end{equation}

\noindent
If we assume \(T_{u,t,j}\) is a binomial random variable, we have:

\begin{equation}
  T_{u,t,j} \sim \text{Binomial}(\pi_j,T_{u,t,j} + T_{f,t,j}).
  \label{eq:air-expand-binomial}
\end{equation}

\noindent
Here, \(\pi_j\) represents the probability that a tagged fish in
spawning tributary \(j\) was outside of the survey flight reach at the
time of the aerial telemetry flight. If we rearrange \(\pi_j\) to be on
the odds scale, we have:

\begin{equation}
  \psi_j=\frac{\pi_j}{1-\pi_j}.
  \label{eq:air-expand-odds}
\end{equation}

\noindent
Estimated expansion factors are shown in Table
\ref{tab:expansion-table}. The odds value \(\psi_j\) can be substitiuted
for the division term in Equation \eqref{eq:air-expand2} which gives:

\begin{equation}
  A_{u,t,j} = A_{f,t,j} \psi_j.
  \label{eq:air-expand3}
\end{equation}

\noindent
To obtain the total number of fish that would have been counted had the
entire subdrainage been flown (\(\hat{A}_{t,j}\)), we can simply sum the
components:

\begin{equation}
  \hat{A}_{t,j} = A_{f,t,j} + A_{u,t,j}.
  \label{eq:air-expand4}
\end{equation}

\noindent
Substistituion of Equation \eqref{eq:air-expand3} into
\eqref{eq:air-expand4} and factoring gives the estimator:

\begin{equation}
  \hat{A}_{t,j}=A_{f,t,j}(1 + \psi_j).
  \label{eq:air-expand-final}
\end{equation}

\paragraph{Aggregate harvest}\label{aggregate-harvest}

\paragraph{Age composition}\label{age-composition}

\subsection{Simulation-estimation
analysis}\label{simulation-estimation-analysis}

\subsubsection{Operating model: Biological
submodel}\label{operating-model-biological-submodel}

Given that the state-space model is a much more natural model of this
system (which has intrinsic time series properties) than the
regression-based versions, it will be used as the foundation operating
model (i.e., state-generating model). The biological submodel will be
more complex than the most complex estimation model -- namely in regards
to the maturity schedule, which will have a modest level of substock
variability but with highly correlated brood year variability. In order
to serve as the state-generating model for the simulation, the
state-space model needs only to be populated with true parameters,
initial states, and a harvest control rule. I will use a fixed
escapement policy with implementation error to ensure the data time
series are generated with patterns consistent with realistic
exploitation patterns (the policy will not be updated as more data are
available). I will generate \(n_j\) = 12 substocks with different
parameters \(U_{\text{MSY},j}\) and \(S_{\text{MSY},j}\) which (as a
starting point) will be informed from random draws from the joint
posterior distribution of 13 substocks from the Kuskokwim River
drainage.

\subsubsection{Operating model: Observation
submodel}\label{operating-model-observation-submodel}

For a given set of simulated true states, a set of observed states
(\(S_{obs,t,j}\), \(H_{obs,t}\), \(q_{obs,t,a}\)) will be generated by
adding sampling error to each year, which will represent the value that
would be observed if the sampling project operated that year.
Observation errors in escapement and harvest estimates will be lognormal
and multinomial for the age composition, as assumed in the state-space
estimation model. Frequency of sampling on each substock (i.e.,
simulated data collection) will be set to approximately mimic the
Kuskokwim River historical monitoring program. Approximately half of the
substocks will have age composition data sampled in the same years as
escapement, and aggregate harvest (\(H_{obs,t}\)) will be available
every year in each simulation.

\subsection{Metrics of model
performance}\label{metrics-of-model-performance}

\section{Results}\label{results}

I found some stuff.

\section{Discussion}\label{discussion}

Here's what it means.

\clearpage

\begin{verbatim}
## Warning: package 'dplyr' was built under R version 3.5.1
\end{verbatim}

\begin{verbatim}
## 
## Attaching package: 'dplyr'
\end{verbatim}

\begin{verbatim}
## The following objects are masked from 'package:stats':
## 
##     filter, lag
\end{verbatim}

\begin{verbatim}
## The following objects are masked from 'package:base':
## 
##     intersect, setdiff, setequal, union
\end{verbatim}

\begin{verbatim}
## Warning: package 'kableExtra' was built under R version 3.5.1
\end{verbatim}

\begin{table}

\caption{\label{tab:ch4-notation-table}Description of the various indices used in the description of the state-space model. $n_t$ is the number of years observed for the most data-rich stock.}
\centering
\begin{tabular}[t]{l>{\raggedright\arraybackslash}p{25em}>{\raggedright\arraybackslash}p{10em}}
\toprule
\textbf{Index} & \textbf{Meaning} & \textbf{Dimensions}\\
\midrule
$y$ & Brood year index; year in which fish were spawned & $n_y=n_t + n_a - 1$\\
$t$ & Calendar year index; year in which observations are made & $n_t$\\
$j$ & Substock index & $n_j$\\
$a$ & Age index; $a=1$ is the first age; $a=n_a$ is the last age & $n_a$\\
$a_{min}$ & The first age recruits can mature & 1\\
$a_{max}$ & The last age recruits can mature & 1\\
\bottomrule
\end{tabular}
\end{table}

\clearpage

\begin{figure}
  \centering
  \includegraphics{img/Ch4/trade-off-plot.png}
  \caption{Visualization of how different types of hetergeneity in substock productivity and size influence the shape of trade-offs in mixed-stock salmon fisheries. Solid black likes are the case where stock types are split evenly among large/small and productive/unproductive stocks. Dotted black likes are the case where all small stocks are productive and all large stocks are unproductive, and dashed lines are the opposite (i.e., all big stocks are productive). (\textit{a}) Equilibrium aggregate harvest and proportion of substocks overfished plotted against the exploitation rate (\textit{b}) value of the biodiversity objective (0 = all stocks overfished) plotted against the value of harvest (the long term proportion of the aggregate MSY attained). Notice that when all big stocks are productive (dashed lines), the trade-off is steeper, i.e., more harvest must be sacrificed in order to ensure a greater fraction of substocks are not overfished. }
  \label{fig:trade-off-plot}
\end{figure}

\clearpage

\begin{figure}
  \centering
  \includegraphics{img/Ch4/obs-freq.jpg}
  \caption{The frequency of escapement sampling for each substock sampled in the Kuskokwim River. Black points indicate years that were sampled for substocks monitored with a weir and grey points indicate years sampled for substocks monitored with aerial surveys. The vertical black line shows a break where > 50\% of the years were monitored for a stock.}
  \label{fig:obs-freq}
\end{figure}

\chapter{Conclusions}\label{ch5}

This chapter contains my thoughts on the topic of the dissertation. What
was found, what will be useful to use in the future, what should be
looked at in more detail?

\chapter*{Appendix A}\label{appendix-a}
\addcontentsline{toc}{chapter}{Appendix A}

\noindent
This appendix contains the necessary code to perform two of the main
parts of the run timing forecast model approach in Chapter \ref{ch2}.

\section*{Forecast Cross-Validation}\label{forecast-cross-validation}
\addcontentsline{toc}{section}{Forecast Cross-Validation}

\subsubsection*{Function Name}\label{function-name}
\addcontentsline{toc}{subsubsection}{Function Name}

\texttt{forecast.CV}

\subsubsection*{Purpose}\label{purpose}
\addcontentsline{toc}{subsubsection}{Purpose}

\subsubsection*{Arguments}\label{arguments}
\addcontentsline{toc}{subsubsection}{Arguments}

\noindent
\textbf{Arguments:}

\begin{enumerate}
\def\labelenumi{\arabic{enumi}.}
\tightlist
\item
  \texttt{x}: a vector containing the time series of the x-variable
\item
  \texttt{y}: a vector containing the time series of the y-variable
\item
  \texttt{start.ind}: the index to start the forecast cross-validation
  (e.g., 10 would train to 10 years and start forecasting in the
  \(11^{\text{th}}\), then continue until present).
\item
  \texttt{na.rm}: logical; do you wish to remove \texttt{NA}
  observations before calculating summary statistics?
\item
  \texttt{include.last.year.in.scores}: logical; do you wish to have the
  last year of \texttt{y} to influence the cross-validation score?
\end{enumerate}

\subsubsection*{Psuedocode}\label{psuedocode}
\addcontentsline{toc}{subsubsection}{Psuedocode}

\subsubsection*{Source Code}\label{source-code}
\addcontentsline{toc}{subsubsection}{Source Code}

\begin{singlespace}
\begin{verbatim}
forecast.CV = function(x, y, start.ind, na.rm = F, include.last.year.in.scores = T) {  
  # total number of observed pairs
  n = length(x)
  # validation end years
  val.end = start.ind:(n-1)
  n.val = length(val.end)
  # containers
  error = numeric(n.val)
  abs.error = numeric(n.val)
  pred.se = numeric(n.val)
  # containers for training data
  train.x = list()
  train.y = list()
  # containers for validation data
  val.x = numeric(n.val)
  val.y = numeric(n.val)
  pred.val.y = numeric(n.val)
  for (i in 1:n.val) {
    # indices to train and validate over
    train.ind = 1:val.end[i]
    val.ind = max(train.ind) + 1
    # store the training data
    train.x[[i]] = x[train.ind]
    train.y[[i]] = y[train.ind]
    # store the validation data
    val.x[i] = x[val.ind]
    val.y[i] = y[val.ind]
    # fit model to training data
    temp.x = train.x[[i]]; temp.y = train.y[[i]]
    fit = lm(temp.y ~ temp.x)
    sig = summary(fit)$sigma
    # forecast
    pred.val.y[i] = predict(fit, newdata = data.frame(temp.x = val.x[i]))
    # statistics
    error[i] = val.y[i] - pred.val.y[i]
    abs.error[i] = abs(error[i])
  }
  
  if (!include.last.year.in.scores) {
    error[n.val] = NA
    abs.error[n.val] = NA
  } 
  
  # return output
  output = list(error = error, abs.error = abs.error, mae = mean(abs.error, na.rm = na.rm))
  return(output)
}
\end{verbatim}
\end{singlespace}

\section*{Sliding Climate-Window}\label{sliding-climate-window}
\addcontentsline{toc}{section}{Sliding Climate-Window}

\setlength{\parskip}{6pt plus 2pt minus 1pt}

\bibliography{cites-without-doi.bib,cites-with-doi.bib}


\end{document}
